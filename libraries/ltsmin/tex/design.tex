
\chapter{Design Rationale}

\section{Introduction}

\begin{enumerate}
\item The LTSmin toolset can be used in two ways:
\begin{enumerate}
\item It can be used as a library that provides functionality and data types.
\item The LTSmin toolset provides language independent tools, such as the bisimulation reduction tools.
\end{enumerate}
\item The LTSmin toolset provides a number of building blocks:
\begin{enumerate}
\item A language independent I/O library.
\item Data value synchronization for MPI.
\item State storage structures:
\begin{enumerate}
\item Sets of states using decision diagrams, optionally including transition relations.
\item Sets of states using tree compression.
\end{enumerate}
\item Model optimizers:
\begin{enumerate}
\item PINS based next state caching.
\item PINS based partial order reduction.
\item PINS based state representation mangling.
\end{enumerate}
\end{enumerate}
\end{enumerate}


\section{Model optimizers}

The problem with model optimizers is that they influence how the interaction
between language module and the LTSmin library works out.

\subsection{next state caching}

This is merely a speedup. It does not influence interaction.

\subsection{partial order reduction}

Partial order reduction influences how traces are reported back to the language
module. If the partial order reduction merges steps then these steps have to be unmerged
when the trace is reported. Or else we must restrict to partial order methods that
work by selecting subsets of the possible transitions.

\subsection{state representation mangling}

The major problem with state mangling and demangling is that the (de)mangling needs the dependency matrices.
A big question is if the LTS files should be written in the original representation or in the mangled representation.

In principle the mangled representation should lead to better compression and better performance of generic tools.
But native tools should still be able to read and understand the mangled representation files and that means
the capability to unmangle has to be in the files. This means that we need to either store dependency matrices,
which were used to compute the mangling, or we should restrict file mangling to permutation and add tree compression.
In that case more advanced state manglers could be implemented as state-set to state-set wrappers rather than
pins to pins wrappers.


\subsection{Other questions}

\subsubsection{What is the best way of specifying a permutation?}

We have two options:
\begin{enumerate}
\item Explicitly as a list of integers.
\item Implicitly by giving every variable a name and using a different variable order
in the signature of the LTS.
\item Implicitly by specifying a tree compression order/structure on the state labels
and edges and keeping the original LTS type.
\item Both of the previous.
\end{enumerate}



\subsubsection{Do we need to store information twice?}

If we store full states then computed state labels can in principle be recomputed.
It would be possible to store them as an ETF map rather than as an explicit list.

\section{Management of data values.}

All data type functionality is provided by the languages modules.
The tool level itself can only manage integers, chunks and strings.
Thus, every other data type has to be serialized to a chunk
and/or pretty printed to a string.

While this approach makes the tool design simpler, it does add cost
in time and memory: (de)serialization is expensive in time and the
serialized form often takes up a lot more memory than the real data
type. For example, it takes time to print/parse ATerms. Moreover,
ATerm use maximal sharing to reduce the memory footprint which is not
possible in the serialized form.

Because of the time cost, we try to avoid (de)serialization by
keeping a global integer to value mapping, so we can send integers
instead of chunks. Because of the memory cost, duplication of
serialized values should be avoided. That is, tools that start language
modules should use the language modules to store values in native form
and the file format should store all values for each type once.

This requirement is somewhat at odds with the fact that the performance
of the symbolic data structures depends on the integers used.
A variable with integers $1,2,3,4$ performs better than one with
$1,5,45,81$, simply because the first set can be represented with less bits.
This is solved by introducing a global map for each variable that maps
the used indices to a consecutive range.


\section{Conventions}

In principle, we want to implement language independent tools.
However, some tools need to know things. For branching bisimulation
we need to know which transitions are invisible. For lumping we
need to know how to add rates. In other cases, having a convention
improves the usability or the performance a lot.

\subsection{Reserved type names}

Boolean and 32 bit natural seem obvious. Rate is to be discussed...
\begin{description}
\item[boolean] False is represented by $0$ and true is represented by $1$. No chunk table used.
\item[nat32] Unsigned 32 bit integer. No chunk table needed.
\item[num32] The 32 bit unsigned numerator of a rate.
\item[denom32] The 32 bit unsigned denominator of a rate.
\item[rate32] Every chunk contains 2 unsigned 32 bit integers in network byte order, representing a rational number.
\item[float] A 32bit floating point number. Could be used as a rate.
\end{description}

\subsection{Reserved edge label names}

\begin{description}
\item[action]
\item[rate]
\item[numerator] 
\item[denominator]
\item[visible]
\item[assert]
\item[accept]
\end{description}

Note that if action/visible and numerator/denominator are defined as distinct labels,
we may need implementation tricks such as: \verb+action_visible=action<<1+visible+
and \verb+rate=num*global_lcm_of_denom/denom+ or possibly tree compression for memory efficiency.

\subsection{Reserved state label names}

For edges, one label is good enough the edges are partitioned by group.
For states, we need multiple asserts and accepts because we need the number of dependencies
of every label to be as low as possible.

\begin{description}
\item[assert{\tt .*}] 
\item[accept{\tt .*}] 
\end{description}




