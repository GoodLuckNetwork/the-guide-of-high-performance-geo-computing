\chapter{File Formats}

\section{Preliminaries}

We use the following data types:
\par\noindent\begin{tabularx}{\textwidth}{@{}lX@{}}
Datatype & description
\\
{\tt string} & A string of 7 bit characters encoded as follows. A string of length $N$ is encoded as follows:
first $N$ is written as an int16, followed by the characters of the string as bytes. The 7 bit restriction is
necessary to maintain compatibility with Java.
\\
{\tt varint} & A variable length integer.
\\
{\tt chunk} & The length (as a varint) of the chunk followed by the data.
\end{tabularx}

\section{Containers and compression}

\subsection{Containers}

A container is a way of storing multiple files in a structured way. Simple examples of containers are
directories and ZIP archives. Directories are usually a features of the operating system,
but sometimes they are difficult to use. (MPI-IO doesn't have directory manipulation functions.
NFS doesn't guarantee that if mkdir terminates on one node the directory is visible on all other nodes.)
ZIP archives suffer from the problem that at most one files can be written.
Thus, we developed our own file archive format: GCF. This format has a few features that set it apart from ZIP:
\begin{itemize}
\item Support for parallel and distributed writing. (Multiple workers can open multiple streams each.)
\item Compression is orthogonal to archiving, so every file can use a different compression scheme.
\end{itemize}

\subsection{Compression}

To support compression in an orthogonal way every compressed file starts with a string that describes
the compression method used. Uncompressed files can be provided in two incompatible ways: one can prefix the file with an empty string, which means 
no compression, or one can just start the file without any header.
In GCF archives, every file starts with a header. In a directory, compressed files use headers and uncompressed files do not.
We reserve the extension .dir for uncompressed directories and the extension .dz for compressed directories.

\section{The multi file LTS format, Proposal.}

\subsection{Introduction}

The vector format stores two main tables and several auxiliary tables. The first main table is a table with states and state labels.
The second main table is a table with edges and edge labels. The vector format makes a few assumptions:
\begin{itemize}
\item The set of states is partitioned into segments.
\item The state table and edge table are also partitioned with one partition per segment.
\item Partition $i$ of the state table stores the state and labels belonging to segment $i$.
\item The state labels are always written in index order.
\item States have two representations: vector and index.
\end{itemize}


\begin{table}[tp]
\caption{LTS representation modes.}
\label{mode table}
\begin{tabularx}{\textwidth}{|l|l|X|}
\hline
a & v,- & State representation used in the state table: vector(v) or implicit(-).
\\\hline
b & v , i , s & State representation of the source states of edges: vector(v), index(i) or segment and offset(s).
\\\hline
c & v , i , s & State representation of the destination states of edges: vector(v), index(i) or segment and offset(s).
\\\hline
\end{tabularx}
\end{table}

The layout of the main tables is determined by a sequence of three characters: {\em abc}.
For an overview see Table \ref{mode table}.
In general v means that the whole vector is stored, s means that both segment and offset are stored,
i means that an index is used and ${-}$ means that the record number is
the index. The most space consuming format is vvv, which means that every state is represented by the full state vector.
This format is necessary if we do not know any state index. However, even if we do not maintain a state index (E.g. \verb+spec2lts-grey --state=vset+) we still know the index of the source state of a transition: the position of a state in the state label file defines one
and we can use this for writing edges. Thus, we get the viv format.

A vector format file, stores the states and transitions
in $W$ partitions each. For each partition, a list of source states and state labels is stored
as one file per state slot and state label. This defines a segmented state numbering.
In a distributed application that does not maintain indices (E.g. on-disk generation) a format such as viv is sensible.
If indices are known then a format such as vsi (state labels written by state owner, edges written by destination owner which means that the source
segment has to be recorded) makes sense. After bisimulation reduction, state information is not very meaningful anymore,
so we can go to ${-}$si or ${-}$is for reduced LTSs.

Converting viv to vsi needs a lot of memory: the entire set of states needs to be stored in memory.
By doing the conversion in two passes, we can reduce the memory requirement. In pass 1, we convert
from viv (states+outgoing edges) to vsv (states+incoming edges), which uses no extra memory. In the second pass we convert from
vsv to vsi, which requires just the states of one segment in memory.

\subsection{Data streams of the vector format, version 1.0}

\par\noindent\begin{tabularx}{\textwidth}{llX}
{\bf file} & {\bf mode} & {\bf description}
\\
\verb+info+ & ??? & Stores the meta information of the LTS.
\\
\verb+SV-+$i$\verb+-+$k$ & v?? & Contains the State Vector value of slot $k$ for states in partition $i$.
\\
\verb+SL-+$i$\verb+-+$k$ & ??? & Contains the State Label value of label $k$ for states in partition $i$.
State labels, which are also state slots are stored under the slot number.
\\
\verb+ES-+$i$\verb+-+$k$ & ?v? & Contains the Edge Source value of slot $k$ for edges in partition $i$.
\\
\verb+ES-+$i$\verb+-seg+ & ?s? &Contains the Edge Source segment for edges in partition $i$.
\\
\verb+ES-+$i$\verb+-ofs+ & ?[is]? & Contains the Edge Source state number for edges in partition $i$.
\\
\verb+ED-+$i$\verb+-+$k$ & ??v & Contains the Edge Destination value of slot $k$ for edges in partition $i$.
\\
\verb+ED-+$i$\verb+-seg+ & ??s & Contains the Edge Destination segment for transitions in partition $i$.
\\
\verb+ED-+$i$\verb+-ofs+ & ??[is] & Contains the Edge Destination state number for transitions in partition $i$.
\\
\verb+EL-+$i$\verb+-+$k$ & ??? & Contain the Edge Label value of label $k$ for the edges in partition $i$.
\\
\verb+CT-+$k$ & ??? & Stores serialized values of the sort $k$.
\end{tabularx}

By using numbers instead of names, we avoid problems with names that cannot be file names.
(E.g. a probabilistic LTS where non-deterministic choice is interleaved
 with probabilistic choice might have type "action\/probability".)

\subsection{The info stream of the vector format}

\subsubsection{top level}

\par\noindent\begin{tabularx}{\textwidth}{lX}
{\tt string} & A string describing the lts format and version. (E.g. "viv 1.0").
\\
{\tt string} & Comment string.
\\
{\tt int32} & The segment count.
\\
{\tt chunk} & Description of the initial state.
\\
{\tt chunk} & A serialization of the LTS type.
\\
{\tt chunk} & The state and transition counts of each partition. For viv the transition counts are out counts, for vsi the counts are in counts.
\\
{\tt chunk} & A serialization of the compression tree. If the chunk size is 0 then
tree compression was not used.
\end{tabularx}

\subsubsection{The initial state}

\par\noindent\begin{tabularx}{\textwidth}{lX}
{\tt int32} & The root segment. Any value that is not in the range [0,segment count) means undefined.
\\
{\tt int32} & The root offset. Has no meaning if the root segment is undefined.
\\
{\tt int32} & The length $l$ of the initial state vector. (0 means no initial state vector).
\\
{\tt int32}${}^l$ & The initial state.
\end{tabularx}


\subsubsection{The serialization of LTS type}

\par\noindent\begin{tabularx}{\textwidth}{lX}
{\tt string} & A string describing the version: "lts signature 1.0".
\\
{\tt int32} & The length of the state vector: vlen.
\\
({\tt string} {\tt int32})${}^{\rm vlen}$ & For each state vector slot a string describing the name and an integer that describes the type.
An empty string means undefined.
\\
{\tt int32} & The number of defined state labels: sl.
\\
({\tt string} {\tt int32})${}^{\rm sl}$ & For each defined state label a string describing the name and an integer that describes the type.
\\
{\tt int32} & The number of edge labels: el.
\\
({\tt string} {\tt int32})${}^{\rm el}$ & For each edge label a string describing the name and an integer that describes the type.
\\
{\tt int32} & The number of types: T.
\\
{\tt string}${}^{\rm T}$ & For each type a string describing the name of the type.
\end{tabularx}

\subsubsection{The state and transition count chunk}

If there are $p$ partitions then this chunk contains the $p$ state counts as {\tt int32} followed by
the $p$ transition counts as {\tt int32}.

\subsubsection{tree compression chunk}

This is a placeholder for a tree compression extension, which is being discussed
at the moment.

\section{The DIRectory format}

The DIR format stores a labeled transition system as multiple streams in a container.
The original tools can deal with one type of container only: an uncompressed directory.
The LTSmin tools can deal with this format in any supported container.

\subsection{data streams}

The DIR format specifies the following data streams:
\par\noindent\begin{tabularx}{\textwidth}{@{}lX@{}}
stream & contents
\\
{\tt info} & global information, see section \ref{info format} for details.
\\
\begin{tabular}[t]{@{}l@{}}
{\tt src-$i$-$j$}\\
{\tt label-$i$-$j$}\\
{\tt dest-$i$-$j$}
\end{tabular}
& the source, label and destination of transitions from segment $i$ to segment $j$.
All values are stored as {\tt int32}.
\\
\begin{tabular}[t]{@{}l@{}}
{\tt parent-seg-$i$}\\
{\tt parent-ofs-$i$}\\
{\tt parent-lbl-$i$}
\end{tabular}
& These files store reverse transitions which form a spanning tree for the state space.
The offset and the label are stored as {\tt int32}, the segment is stored as {\tt int8}.
The parent of the root is the root itself.
\\
\begin{tabular}[t]{@{}l@{}}
{\tt TermDB}\\
{\tt TermDB.gz}
\end{tabular}
& Mapping from {\tt ATerm}'s to integers. This is a simple text file with lines counting form $0$, the
term which maps to $i$ is printed on line $i$ of the file. The mapping is used for both labels and vector terms.
Precisely one of these files must exist. If the {\tt .gz} extension is used then the file is gzip compressed.
\\
{\tt target-$i$}&
ASCII file. Each line consists of a target state (number) followed by the actions (strings) which are 
enabled by that state.
A target state which is not followed by any action is a deadlock state.
\\
\begin{tabular}[t]{@{}l@{}}
{\tt node-$i$-$j$}\\
{\tt node-$i$}
\end{tabular}
&
These files store the node databases as lists of {\tt int32} pairs.
If the database for node $i$ is a bottom database then it is stored in
{\tt node-$i$}. Otherwise, segment $j$ of that database is stored in {\tt node-$i$-$j$}.
\end{tabularx}

\subsection{contents of the info stream}
\label{info format}
The contents of the info file are as follows:


\par\noindent\begin{tabularx}{\textwidth}{lX}
{\tt int32} & A version number, which allows testing if the directory is in the correct format.
The current version is 31.
\\
{\tt string} & A short description of the transition system.
\\
{\tt int32} & The number of segments $N$.
\\
{\tt int32} & The segment of the initial state.
\\
{\tt int32} & The offset of the initial state.
\\
{\tt int32} & The number of entries in the {\tt TermDB} file.
\\
{\tt int32} & The index of the silent transition.
\\
{\tt int32} & The number of top databases $K$.
\\
{\tt int32} & The number of states in segment $0$.
\\
&$\vdots$
\\
{\tt int32} & The number of states in segment $N-1$.
\\
{\tt int32} & The number of transitions from segment $0$ to segment $0$.
\\
{\tt int32} & The number of transitions from segment $0$ to segment $1$.
\\
&$\vdots$
\\
{\tt int32} & The number of transitions from segment $N-1$ to segment $N-1$.
\end{tabularx}


